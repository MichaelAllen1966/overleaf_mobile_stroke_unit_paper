\section*{Abstract}

\textbf{Background}: Intravenous thrombolysis (IVT) and mechanical thrombectomy (MT) are two well established treatments for reduction or removal of clots causing the stroke, reducing disability caused by stroke, but both lose effectiveness in the hours following a stroke. Mobile stroke units (MSUs), which enable diagnosis and IVT treatment of stroke on-scene, have been proposed as a way of improving time to either treatment.

\textbf{Key objectives}:

\begin{enumerate}
    \item Model the likely effect of MSUs on time to IVT and MT across all of England.
    \item Model the likely effect of MSUs on outcomes (proportion of patients mRS 0-2, and the utility outcome) across all of England.
    \item Perform a comprehensive analysis of the effect of changing process times for usual care and MSUs.
    \item Investigate how the number of MSU base locations affects IVT/MT times and resulting outcomes.
\end{enumerate}

\textbf{Methods}: We used modelling of times to treatment, and outcomes. Times to IVT and MT were estimated across 32,843 Lower Super Output Areas (LSOAs) in England, taking into account process times and travel times (which vary by LSOA). Outcomes were predicted based on times to IVT and MT; we report outcomes as utility or the proportion of patients with mRS 0-2 at 6 months.

\textbf{Results}: Across a range of likely process time scenarios, we found net benefit, if a patient received MSU care in place of usual care, was likely to be an increase of 0.015-0.03 in either utility or the proportion of patients with mRS 0-2 at 6 months. The benefit to nLVO was around 0.015 in either measure. The benefit to LVO was larger (around 0.035 in both measures), with the majority of this benefit coming from transferring LVOs directly to an MT-capable centre. The net benefit varied by patient location, with some benefit being up to 0.04 improvement in either utility or the proportion of patients with mRS 0-2, however some regions would have worse net outcomes with use of MSU. The net benefit of MSUs diminished with distance from the MSU base, though there was a halo effect for the benefit for LVOs around the MT-capable centres, with LVO patients who would otherwise first attend an IVT-only centre under usual care benefiting from direct transfer to an MT-capable centre under MSU care. The benefit of MSUs was critically dependent on rapid dispatch of the MSU and efficient IVT on-scene.

\textbf{Conclusions}: Overall we found a relatively small benefit from MSUs across the country, though some areas have greater benefit. It is possible that more selective targeting of who to provide an MSU could help maximise benefit. A significant part of their potential benefit is derived from avoiding transfers for patients suitable for MT, reducing time to MT significantly. Maximising benefit from MSUs is critically dependent on rapid dispatch, and fast on-scene IVT.