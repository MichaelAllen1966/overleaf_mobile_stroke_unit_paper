\section*{Abstract}

\textbf{Background}: Intravenous thrombolysis (IVT) and mechanical thrombectomy (MT) are two well established emergency reperfusion treatments for clots causing stroke. Both reduce disability but lose effectiveness in the hours following symptom onset. Mobile stroke units (MSUs), equipped with brain imaging capability, enable pre-hospital diagnosis and IVT treatment on-scene and may improve access to regional MT centres, have been proposed as a way of improving time to treatment.

\textbf{Key objectives}:

The primary objective of the study was to model the likely effect of MSUs on time to IVT and MT, and resulting clinical outcomes (especially the ability to live independently, modified Rankin Score 0-2) across all of England assuming no resource restrictions during full deployment.

\textbf{Methods}: We used modelling of times to treatment, and outcomes. Times to IVT and MT were estimated across Lower Super Output Areas (LSOAs) in England, taking into account process times and predicted travel times. Outcomes were predicted based on times to IVT and MT; we report outcomes as utility or the proportion of patients able to live independently at 6 months.

\textbf{Results}: For every 100 patients suitable for IVT or MT, there will likely be 1-3 more people who can live independently under MSU care because of earlier treatment (the range depends on how efficient MSU care is compared to usual care). The benefit comes from both earlier IVT and the direct transfer of patients likely to benefit from MT to their closest MT-centre (avoiding inter-hospital transfers that may be used in usual care). If, as is likely, about 1 in 5 stroke patients are suitable candidates for IVT or MT, the MSU would need to attend approximately 250 stroke patients for every one extra independent-living outcome. If about half of the patients whom the MSU is dispatched to are actual strokes (the others being stroke mimics), the MSU would need to attend approximately 500 patients for every one extra independent-living outcome. Some areas, furthest from where MSUs are based, will receive no benefit from MSUs, other areas may have up to 4 additional independent-living outcomes for every 100 patients suitable for IVT or MT. Quick MSU dispatch and fast on-scene treatment are crucial to achieving the benefit of MSUs, otherwise use of MSUs may have no overall benefit, or worse outcomes, than usual care. The above benefits do not include any other possible benefits unrelated to earlier IVT or MT.

\textbf{Conclusions}: This study suggests that the overall benefit of MSU care is likely to be modest. Selective use of MSUs in specific areas might be more effective than widespread implementation. Rapid dispatch, fast on-scene treatment of patients, and careful selection of which patients to dispatch the MSU to (by location and confidence in that person being a confirmed stroke patient), are all critical for achieving benefit from MSU care. MSUs should not be seen as an alternative to optimising day-to-day emergency stroke systems.