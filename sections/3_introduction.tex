\section{Introduction}

% What is the problem (stroke eburden)

Stroke remains one of the top three global causes of death and disability \cite{feigin_global_2021}. Despite reductions in age-standardised rates of stroke, ageing populations are driving an increase in the absolute number of strokes \cite{feigin_global_2021}. Across Europe, in 2017, stroke was found to cost healthcare systems \texteuro 27 billion, or 1.7\% of health expenditure \cite{luengo-fernandez_economic_2020}.

Intravenous thrombolysis (IVT) and mechanical thrombectomy (MT) are two well established treatments for reduction or removal of clots causing the stroke, reducing disability caused by stroke, but both lose effectiveness in the hours following a stroke \cite{emberson_effect_2014, fransen_time_2016}. IVT is suitable for both non-large vessel occlusions (nLVO) and large vessel occlusions (LVO). MT is suitable only for LVO, but has superior efficacy compared to IVT.

% What do we know (see chen_systematic_2022 for review)

As a way of improving time from onset to IVT, mobile stroke units (MSUs) were first proposed in 2003 by Fassbender et al. \cite{fassbender_mobile_2003}. An MSU is a bespoke ambulance that contains the required technology and specialist opinion for diagnosing ischaemic stroke, and making an IVT treatment decision i.e. a computed tomography (CT) scanner, and an in-person or telemedicine review by a stroke specialist \cite{taqui_reduction_2017}. MSUs were first trialled in Homberg, Germany \cite{walter_diagnosis_2012} and have been introduced based around MT-capable centres (comprehensive stroke centres, CSCs)  in other health systems. No health system has yet deployed MSUs at a regional or country wide level.

Fatima \textit{et al}. \cite{fatima_mobile_2020} have published a meta-analysis of MSU trials. They reviewed evidence from a total of 21,297 patients from 11 publications (seven randomized controlled trials and four non-randomized controlled trials including prospective cohort studies). Mean time to IVT was reduced 13 minutes on average, from 75 to 63 minutes. In a pooled analysis they found the odds of a good outcome (modified Rankin Scale, mRS, 0–2 at day 7) were improved 1.46 times by use of an MSU.

In a separate meta-analysis, Chen \textit{et al.} \cite{chen_systematic_2022} reviewed evidence from a total of 22,766 patients from 16 publications. In total 7,682 (33.8\%) were treated in the MSU and 15,084 (66.2\%) with usual care. They found higher use of IVT in the MSU group (37.3\% vs. 27.7\%), and an improvement the proportion of patients with mRS 0-2 at 90 days (66.2\% vs 58.8\%).  The pooled analysis of time metrics indicated a mean reduction of 33 minutes in time to IVT between MSU and usual care.

There is less known about how MSUs affect use, and speed, of MT, where the main advantage for suitable patients would be direct admission to a CSC rather than a local hospital which cannot provide MT, when a secondary transfer would be required. A review of available evidence suggests there is an evidence gap in MSUs and MT, as the MSU were dispatched from CSCs, into populations which would have been directly admitted under usual care \cite{navi_mobile_2022}. In Berlin, Ebenger et al. \cite{ebinger_association_2021} found MT rate and speed was essentially unchanged (13.8\% vs 14.2\%, and median dispatch to MT 137 mins vs 125 mins, for MSU and usual care). In the BEST Study \cite{grotta_prospective_2021} median time from 911 alert to MT was 141 and 132 minutes for MSU and usual care. The BEST-MSU substudy \cite{czap_abstract_2022} was a subset of the BEST study, for IVT-eligible stroke patients with LVOs MT rates were 75\% and 83\% for MSU and usual care,  with median onset-to-puncture times of 141 min and 132 minutes.

% What do we not know (Impact in England, Demographics, Geography)

MSU trials have generally been in metropolitan areas, where the MSU can respond rapidly to suspected stroke and travel times are relatively short. In order to predict the benefit of MSUs in a broader geographic area Holodinsky et al. \cite{holodinsky_jessalyn_k_what_2020} used modelling to predict the benefit, as proportion of patients with an outcome of mRS 0-2, from IVT when using MSUs to cover larger distances (up to 4.5 hours travel for the MSU). They found that the effect of MSUs was likely to be  minimal in the regions close to the stroke centre which was the MSU base. They predicted no more than 1 percentage point increase in the proportion of patients with an outcome of mRS 0-2, but this could rise to about a 2 percentage point increase in a halo further away from the stroke centre where the MSU made the difference between receiving IVT or not.

% What are our aims here? 

Our work aimed to model the effect of MSUs on likely times to both IVT and MT (including modelling inter-hospital transfers when required), and to estimate outcomes both as the proportion mRS 0-2 at 6 months, and their corresponding health utility. We modelled these effects across  all of England (population 58 million in 2023) performing a detailed geographic analysis at the Lower Super Output Areas (LSOA) level. We performed a comprehensive analysis of the effect of changing process times for usual care and MSU care. We also investigated how the number of MSU base locations affects IVT/MT times and resulting patient outcomes.