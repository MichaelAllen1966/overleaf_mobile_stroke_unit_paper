\section{Introduction}

% What is the problem (stroke eburden)

Stroke remains one of the top three global causes of death and disability \cite{feigin_global_2021}. Despite reductions in age-standardised rates of stroke, ageing populations are driving an increase in the absolute number of strokes \cite{feigin_global_2021}. Across Europe, in 2017, stroke was found to cost healthcare systems \texteuro 27 billion, or 1.7\% of health expenditure \cite{luengo-fernandez_economic_2020}.

Intravenous thrombolysis (IVT) and mechanical thrombectomy (MT) are two well established treatments for reduction or removal of clots causing the stroke, reducing disability caused by stroke, but both lose effectiveness in the hours following a stroke \cite{emberson_effect_2014, fransen_time_2016}. IVT is suitable for both non-large vessel occlusions (nLVO) and large vessel occlusions (LVO). MT is suitable only for LVO, but has superior efficacy compared to IVT.

% What do we know (see chen_systematic_2022 for review)

As a way of improving time from onset to thrombolysis, mobile stroke units were first proposed in 2003 by Fassbender et al. \cite{fassbender_mobile_2003}. A mobile stroke unit is a specialised ambulance that contains the required tools for diagnosing an ischaemic stroke, and treating the patient with thrombolysis. They require a CT scanner, and specialised staff who can perform the scan (though review of the scan images, and other specialised consultation, may be performed remotely \cite{taqui_reduction_2017}). Mobile stroke units were first trialled in Homberg, Germany\cite{walter_diagnosis_2012}.

Fatima \textit{et al}. \cite{fatima_mobile_2020} have published a meta-analysis of mobile stroke unit trials. They reviewed evidence from a total of 21,297 patients from 11 publications (seven randomized controlled trials and four non-randomized controlled trials including prospective cohort studies). They found mean time to IVT reduced 13 minutes on average, from 75 to 63 minutes. In a pooled analysis they found the odds of a good outcome (modified Rankin scale 0–2 at day 7) were improved 1.46 times by use of a mobile stroke unit, though there was no change in odds of death.

In a separate meta-analysis Chen \textit{et al.} \cite{chen_systematic_2022}. A total of 22,766 patients from 16 publications were included. In total 7,682 (n = 33.8\%) were treated in the MSU and 15,084 (n = 66.2\%) with conventional care. They found higher use of thrombolysis in the MSU group (37.3\% vs. 27.7\%), and an improvement the proportion of patients with mRS0-2 at 90 days (66.2\% vs 58.8\%).  The pooled analysis of time metrics indicated a mean reduction of 33 minutes in time-to-therapy between MSU and usual care.

There is less known about how mobile stroke units affect use, and speed, of MT. A review of available evidence suggests there is an evidence gap in MSUs and MT \cite{navi_mobile_2022}. In Berlin, Ebenger et al. \cite{ebinger_association_2021} found MT rate and speed was essentially unchanged (13.8\% vs 14.2\%, and median dispatch to MT 137 mins vs 125 mins, for MSU and usual care). In the BEST Study \cite{grotta_prospective_2021} median time from 911 alert to MT was 141 and 132 minutes for MSU and usual care. The BEST-MSU substudy \cite{czap_abstract_2022} was a subset of the BEST MSU study, for IVT-eligible stroke patients with LVOs on CT and/or CTA. MT rates were 75\% and 83\% for MSU and usual care,  with median onset-to-puncture times of 141 min and 132 minutes.

% What do we not know (Impact in England, Demographics, Geography)

Mobile stroke unit trials have generally been in metropolitan areas, where the mobile stroke unit can respond rapidly to suspected stroke. In order to predict the benefit of mobile stroke units in a broader geographic area Holodinsky et al. \cite{holodinsky_jessalyn_k_what_2020} used modelling to predict the benefit, as proportion of patients with an outcome of mRS 0-2, from IVT when using mobile stroke units. They found that the effect of mobile stroke units was likely to be generally minimal in the regions close to the stroke centre which was the MSU base. They predicted no more than 0.01 increase in the proportion of patients with an an outcome of mRS 0-2, but this could rise to about a 0.02 increase in a halo further away from the stroke centre where the MSU made the difference between receiving IVT or not receiving IVT.

% What are our aims here? 

Our aims in this work were to model the effect of MSUs on likely times to both IVT and MT (including modelling inter-hopsital transfers when required), and to estimate outcomes both as the proportion mRS 0-2 at 6 months, and their corresponding utility. We aimed to model the across all of England in order to perform a detailed geographic analysis of all 32,843 small Lower Super Output Areas (LSOAs). We aimed to perform a comprehensive analysis of the effect of changing process times for usual care and MSUs. And we aimed to investigate how the number of MSU base locations affects IVT/MT times and resulting outcomes.