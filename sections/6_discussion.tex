\section{Discussion}

% Include call out, diting, thrombolysis, thrombectomy

% Key findings

Overall we found a relatively marginal predicted benefit of MSUs across the whole country of England. Using likely process timing we found benefit was likely to be an increase of 0.015 to 0.03 in either utility or the proportion of patients with mRS 0-2 at 6 months. The benefit to nLVO was around 0.015 in either measure. The benefit to LVO was larger (around 0.035 in both measures), with the majority of this benefit coming from transferring LVOs directly to a MT-capable centre, avoiding the need for inter-hospital transfers and their associated delays. The benefit from earlier IVT is similar to that modeled by Holodinsky et al. \cite{holodinsky_jessalyn_k_what_2020}. Benefit was larger in the regions within reasonable travel distances of the MSU (similar catchments to those in the clinical trials). 

We found that the benefit of MSUs over usual care was critically dependent on rapid dispatch of the MSU and relatively rapid IVT ($\leq$30 minutes) on-scene. MSUs are therefore not an alternative to careful optimization of dat-to-day operational activities.

The benefit of MSUs diminished with distance from the MSU base, though there was a halo effect for LVOs, with patients benefiting from direct transfer to MT-capable centres without excessive MSU arrival times. The diminishing benefit of MSUs is similar to that observed in clinical trial, where in Berlin the advantage of MSUs over normal care, when considering time to treatment, fell with distance from the MSU base \cite{koch_influence_2016}. MSUs centered in metropolitan areas are therefore not a solution to significantly improving stroke outcomes in more remote locations. If MSUs were based in remote locations there would be challenge of long MSU travel times if the MSU must maximise the number of patients seen each day. Additionally, MSUs will only partially solve the challenge of timely access to MT from remote areas.

We have modelled MSUs  being based at stroke centres. We compared basing MSUs at just comprehensive stroke centres (offering both IVT and MT) or at any type of stroke centres. The difference between these two approaches was marginal - likely because comprehensive stroke centres tend to be sited within large dense population centres, and so capture most population benefit of MSUs. It is possible to site MSUs in other locations, though most benefit will accrue from citing them in or near population centres, which is where stroke centres generally exist. As the number of base locations is increased the possible benefit increases, but with diminishing returns.

A possible difference between our model and real-world use of MSUs is that we assume that for patients within the time window of IVT and MT we assume a similar propensity of clinicians in MSUs and in usual care to give those treatments. This assumption allows us to isolate geographic effects. However, we know different stroke teams vary in their propensity to use IVT \cite{pearn_what_2023}. It is possible that MSUs will be staffed by clinicians more confident in using IVT, and so IVT use may increase not from the altered times to IVT, but by the patient being seen by clinicians more confident in using IVT. This could explain why Chen \textit{et al.} \cite{chen_systematic_2022} found such a significant (34\%) increase in use of IVT in MSU trials; it seems unlikely this change could come just from the speed improvements offered by MSUs.

We present results for patients seen by the MSU compared with normal care. A challenge will be identification of the correct patients to dispatch the MSU. In a 2024 review of Emergency Medical Services dispatcher recognition of stroke \cite{wenstrup_emergency_2024}, Wenstrup \textit{et al}. found sensitivity varied from 17.9\% to 83.0\%. Sensitivity median and interquartile range was 56\% (48\%-63\%). Positive predictive value was reported in 12 papers and ranged from 24.0\% to 87.7\%. PPV median and interquartile range = 46\% (42\%-50\%). Typically therefore half of stroke patients are missed at ambulance dispatch, and only half of suspected stroke patients at dispatch are later confirmed to have a stroke. In one study it was found sensitivity for identifying stroke could be improved, but at the cost of positive predictive value; In a study on the effect of training call handlers \cite{watkins_training_2013}, on 464 patients, sensitivity improved from 63\% to 80\%, but PPV fell from 60.5\% to 39.0\%. Such uncertainty in the sensitivity and positive predictive value of identification of stroke patients makes it difficult to predict how many stroke patients will be seen by MSUs, as that number is limited by both sensitivity of dispatch (where stroke patients are missed) but also by positive predictive value which consumes MSU capacity, risking the MSU not being available as it attends a non-stroke patient.

Our study adds to the evidence base on how MSUs may affect times to MT, especially when used more widely than areas close to comprehensive stroke centres. When the catchment area of MSUs extends out beyond the usual catchment area of comprehensive stroke centres, the MSU captures patients who would otherwise go to a local IVT-only stroke centre and require onward transfer for MT. MSUs therefore have significant potential to improve outcomes for those patients. This effect will be dependent on the MSU using CT-A to identify LVO patients.

\subsection{Study limitations}

Two key limitations have been discussed: Firstly, we isolated the geographic effects of MSUs and do not model how having an expert specialist team may increase IVT use simply by being more experienced, and so more confident, in use of IVT. Secondly, due to large uncertainties around sensitivity and positive predictive value of identification of stroke patients for MSU dispatch, we limit our study to modelling of outcomes of those who are seen by the MSU. Real-world benefit will be diluted by stroke patients being missed, or by MSU capacity not being available when required (especially if capacity is drained by low positive predictive value od suspected stroke). Similarly, for the same reason, we do not model how MSUs may affect emergency stroke admission numbers at hospitals (e.g. by changing effective catchment areas). It is possible that widespread use of MSUs could compromise the ability of some smaller hospitals to still provide IVT themselves (due to loss of experience). We also do not model other potential benefits of MSUs. In haemorrhagic stroke there may be potential to start reducing blood pressure sooner where that would benefit patients. In non-stroke patients it is possible that improved diagnosis by the MSU could help identify the best destination for that patient sooner, or may give more confidence in leaving the patient at home, saving hospital resources.

\subsection{Conclusions}

Overall we found a relatively small benefit from MSUs across the country, though some areas have greater benefit. It is possible that more selective targeting of MSUs use could help maximise benefit. A significant part of their potential benefit is derived from avoiding transfers for patients suitable for MT, reducing time to MT significantly. Maximising benefit from MSUs is critically dependent on rapid dispatch, and fast on-scene IVT. MSUs should not be seen as an alternative to optimizing day-to-day emergency stroke systems.