\section{Discussion}

% Add something about OPTIMIST/SPPEDY

% Key findings

Overall we found a relatively marginal predicted benefit from the widespread adoption of MSUs across the whole of England (58 million population). Using plausible process timings we found the magnitude of benefit was likely to be an increase of 0.015 to 0.03 in health utility or 0.015-0.030 in the proportion of independent patients at 6 months. The benefit to patients with nLVO was around 0.015 in the two measures. The benefit to patients with LVO was larger (around 0.035 in the two measures respectively), with the majority of this benefit coming from the direct conveyance of patients with LVO to a MT-capable centre, avoiding the need for inter-hospital transfers and their associated delays, although there are alternative approaches to also achieve this pathway through equivalent reductions in DIDO times. The benefit from earlier IVT is similar to that modeled by Holodinsky et al. \cite{holodinsky_jessalyn_k_what_2020}. Benefit was larger in the regions within reasonable travel distances of the MSU (similar catchments to those in the clinical trials). 

We have directly modelled the treated population only. The overall net benefit will be diluted by patients the MSU is dispatched to who are either stroke patients who do not receive IVT or MT, or who are not confirmed to have had a stroke. In the treated population for every 100 patients suitable for IVT or MT, there will likely be 1-3 more people who can live independently because of earlier treatment. If only about 1 in 5 stroke patients are suitable candidates for IVT or MT, the MSU would need to attend approximately 250 stroke patients for every one extra independent-living outcome. If about half of the patients whom the MSU is dispatched to are actual strokes (the others being stroke mimics), the MSU would need to attend approximately 500 patients for every one extra independent-living outcome. 

We found that the benefit of MSU care over usual care was critically dependent on rapid dispatch of the MSU and relatively rapid IVT ($\leq$30 minutes) on-scene. MSUs are therefore not an alternative to careful optimisation of day-to-day operational activities.

The benefit of MSU care diminished with distance from the MSU base location, though there was a halo effect for patients with LVO, with patients benefiting from direct transfer to MT-capable centres without excessive MSU arrival times. The diminishing benefit of MSUs is similar to that observed in clinical trials, where in Berlin the advantage of MSU care over usual care, when considering time to treatment, fell with distance from the MSU base location \cite{koch_influence_2016}. MSUs centered in metropolitan areas are therefore not a solution to significantly improving stroke outcomes in more remote locations.However if MSUs were based in remote locations there would be challenge of long MSU travel times if the MSU is to maximise the number of patients seen each day. Additionally, MSUs will only partially solve the challenge of timely access to MT from remote areas. Maximising the net benefit of MSUs may therefore be at the cost of worsening equity of access to emergency stroke reperfusion therapies, as when MSUs are placed in metropolitan areas they improve access to care for those that already have the best access. Therefore care providers should also consider combining MSUs with other approaches which could reduce inequity such as CSC configurations and selection of patients for direct admission during standard ambulance assessment. 

We have compared MSUs to usual care, which will involve inter-hospital transfer for those patients first attending an IVT-only centre. We saw that a significant part of the benefit from MSU for patients with LVO came from avoiding such transfers, with the patient being taken directly to a MT-capable centre. Such benefit may be achieved in other ways such as use clinical symptom scoring for pre-hospital section of patients likely to benefit from MT \cite{perez_de_la_ossa_effect_2022} .

We have modelled MSUs being based at stroke centres. We compared basing MSUs at just comprehensive stroke centres (offering both IVT and MT) or at any type of stroke centres. The difference between these two approaches was marginal - likely because comprehensive stroke centres tend to be sited within large dense population centres, and so capture most population benefit of derivable MSUs. It is possible to site MSUs in other locations such as ambulance stations, though most benefit will accrue from locating them in or near population centres, which is where stroke centres generally exist. As the number of base locations is increased the possible benefit increases, but with diminishing returns. 

A possible difference between our model and real-world use of MSUs is that for patients within the time window of IVT and MT we assume a similar propensity of clinicians in MSUs and in usual care to give those treatments. This assumption allows us to isolate geographic effects. However, we know different stroke teams vary in their propensity to use IVT [25]. It is likely that MSUs will be staffed by clinicians more confident in using IVT, and so IVT use may increase not from the altered times to IVT, but by the patient being seen by clinicians more confident in using IVT. This could partly explain why Chen et al. \cite{chen_systematic_2022} found such a significant (34\%) increase in use of IVT in MSU trials; it seems unlikely this degree of change could come just from the modest speed improvements offered by MSUs.

We present results for patients seen by the MSU compared with usual care. A challenge will be identification of the correct patients to dispatch the MSU. In a 2024 review of Emergency Medical Services dispatcher recognition of stroke \cite{wenstrup_emergency_2024}, Wenstrup \textit{et al}. found sensitivity varied from 17.9\% to 83.0\%. Sensitivity median and interquartile range was 56\% (48\%-63\%). Positive predictive value (PPV) was reported in 12 papers and ranged from 24.0\% to 87.7\% with a median and interquartile range of 46\% (42\%-50\%). Typically therefore half of stroke patients are missed at ambulance dispatch, and only half of suspected stroke patients at dispatch are later confirmed to have a stroke. In one study it was found sensitivity for identifying stroke could be improved, but at the cost of PPV; In a study on the effect of training call handlers \cite{watkins_training_2013}, on 464 patients, sensitivity improved from 63\% to 80\%, but PPV fell from 60.5\% to 39.0\%. Such uncertainty in the sensitivity and PPV of identification of stroke patients makes it difficult to predict how many stroke patients will be seen by MSUs, as that number is limited by both sensitivity of dispatch (where stroke patients are missed) but also by PPV which consumes MSU capacity, risking the MSU not being available as it attends a non-stroke patient. In addition to concerns around identifying patients for MSU dispatch, other implementation concerns have been raised in a qualitative study of clinician views of MSUs \cite{moseley_practitioner_2024}. This includes concerns over how they would be staffed, where they would be based, and whether they will increase or reduce equity of access to emergency stroke care.

Our study adds to the evidence base on how MSUs may affect times to MT, especially when used more widely than areas close to comprehensive stroke centres. When the catchment area of MSUs extends out beyond the usual catchment area of comprehensive stroke centres, the MSU captures patients who would otherwise go to a local IVT-only stroke centre and require onward transfer for MT (effectively an ambulance redirection outcome). MSUs therefore have potential to improve outcomes for those patients. This effect will be dependent on the MSU using CT-A to identify LVO patients and having reliable image interpretation immediately available. Alternative approaches to ambulance redirection are available although use of clinical scale assessment triage, telemedicine or near patient testing are all yet to be proven to improve outcomes at the population level.

\subsection{Study limitations}

Two key limitations have been discussed: Firstly, we isolated the geographic effects of MSUs and do not model how having an expert specialist team may increase IVT use simply by being more experienced, and so more confident, in use of IVT. Secondly, due to large uncertainties around sensitivity and positive predictive value of identification of stroke patients for MSU dispatch, we limit our study to modelling of outcomes of those who are seen by the MSU. Real-world benefit will be diluted by stroke patients being missed, or by MSU capacity not being available when required (especially if capacity is constrained by low PPV of suspected stroke). Similarly, for the same reason, we do not model how MSUs may affect emergency stroke admission numbers at hospitals (e.g. by changing effective catchment areas). It is possible that widespread use of MSUs could compromise the ability of some smaller hospitals to still provide IVT themselves (due to loss of experience). We also do not model other potential benefits of MSUs. For example, in haemorrhagic stroke there may be potential to start reducing blood pressure sooner where that would benefit patients. In non-stroke patients it is possible that improved diagnosis by the MSU could help identify the best destination for that patient sooner, or may give more confidence in leaving the patient at home, saving healthcare resources.

We have not modelled selection of late-presenting patients, or patients with unknown stroke onset time. These patients may be selected for IVT or MT based on advanced imaging. These patients would not be expected to be following the decline in effectiveness of thrombolysis or thrombectomy described in analysis of how time-to-treatment affects outcomes \cite{emberson_effect_2014, fransen_time_2016}.

\subsection{Conclusions}

Overall we found a relatively small benefit from MSUs across the country. Benefit depends on efficiency of dispacth and treatment, and varies with geography. It is possible that more selective targeting of MSUs could help maximise benefit. A significant part of their potential benefit is derived from avoiding transfers for patients suitable for MT, reducing time to MT significantly. Maximising benefit from MSUs is critically dependent on rapid dispatch, and fast on-scene IVT. However, there are other considerations such as resource limitations and implementation challenges, and MSUs should not be seen as an alternative to optimizing day-to-day emergency stroke systems.