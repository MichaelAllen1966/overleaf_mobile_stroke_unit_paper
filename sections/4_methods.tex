\section{Methods}

1.1 Method overview

This work compares the outcome of patients under two treatment delivery models: i) usual care (where the patient is taken to their nearest stroke centre for treatment, with an onwards transfer to a comprehensive stroke centre, CSC, for MT where necessary). and ii) MSU care (where an MSU attends the patient on-scene to provide IVT, with an onwards transfer to a CSC for MT for patients with an LVO). The scope of the modelling includes the stroke patients pathway from stroke onset to time to treatment with IVT and MT, with this time to treatment being translated to the patient \textit{outcome} (dependent also on the stroke type and treatment received). Strokes are defined as either an nLVO or LVO, with their treatment options being IVT (for an nLVO) or IVT followed by MT (for an LVO). Sensitivities around \textit{pathway process} durations are explored using \textit{scenario analysis}, \textit{geographic analysis} explores the variation due to the stroke onset location, and the impact of the \textit{number of MSU base locations} are explored on patient outcomes.

\subsection{Pathway processes modelling}

All code used for pathway processes modelling is available at \url{https://github.com/stroke-modelling/muster2}.

Figure \ref{fig:process} describes the processes included in the three pathways that are modelled (two pathways for usual care depending on which type of stroke centre is nearest to the LSOA, and one pathway for MSU care). \textit{Usual care} is provided by the patient attending their nearest stroke centre, this is either i) a \textit{primary stroke centre}, PSC, providing IVT only, with onward transfer for LVO patients to the comprehensive stroke centre that is nearest to the PSC, for MT, or ii) a \textit{comprehensive stroke centre}, CSC, providing both IVT and MT. The \textit{MSU care} pathway involves the MSU providing on-scene IVT, followed by LVO patients transferred to the closest CSC for MT. The onwards transfer of nLVO patients are not included in the scope of this modelling.

Based on input from three co-production workshops (ref MSU development paper if available as preprint) involving representation form stroke consultants, ambulance staff and patients and public (4-6 from each group at each of three workshops), we designed the model to explore large numbers of possible parameter values. This reflected the uncertainty of how the MSUs would operate and perform if implemented.

Unless specified otherwise, it was assumed that MSUs are located at CSCs. We also assume MSUs are equipped with CTA and can identify LVO patients who will benefit from direct transfer to a CSC for MT.

Geographic analysis was undertaken at Lower Super Output Area (LSOA) level. Travel times from each of the 32,843 LSOAs in England to all hospitals (PSC and CSCs), and travel times between hospitals have been estimated using Open Street Map data, with results calibrated against Google Maps. Travel times have been made available (\url{https://gitlab.com/michaelallen1966/1811_lsoa_to_acute_hospital_travel}).

\begin{figure}[h]
    \centering
    \includegraphics[width=0.85\linewidth]{images/stroke_treatment.jpg}
    \caption{The processes included in the three pathway that are modelled for provision of IVT and MT. Top: Usual care pathway for patients with a PSC closest to their home LSOA, with the PSC providing IVT, followed by the LVO patients having a transfer to the nearest CSC for MT. Middle: Usual care pathway for patients with a CSC closest to their home LSOA, with the CSC providing both IVT and MT. Bottom: The care with MSU pathway, with IVT provided on-scene by the MSU, followed by transfer for LVO patients to the nearest CSC for MT. Process times other than travel times are common for all patients (defined by the scenario). Travel times depend on locations of patient and hospitals, with results calculated for all LSOAs in England.}
    \label{fig:process}
\end{figure}

For each LSOA, times to IVT and MT are calculated by summing all the non-travel process times (common for all patients) and adding required travel times (bespoke for each LSOA location, and for each inter-hospital transfer). The next section will describe how patient outcomes are calculated based on these LSOA-specific times to IVT and MT for usualcare or MSU care. All calculations are performed in Python/NumPy.

\subsection{Outcome modelling}

Detailed methods and code used for modelling outcomes, based on time to IVT and MT is available at \url{https://github.com/samuel-book/stroke_outcome/}.

This section describes how patient outcomes are calculated from time to treatment, based on their stroke type and treatment received (IVT for nLVO and LVO, and MT for LVO).

The stroke patient population were divided into stroke type cohorts, based on vessel occlusion type (LVO and nLVO). Using analysis from reperfusion treatment clinical trials \cite{lees_time_2010, emberson_effect_2014, goyal_endovascular_2016, fransen_time_2016}, we classify each patient's vessel occlusion type based on their National Institutes of Health Stroke Scale (NIHSS) on arrival (NIHSS 0-10 as nLVO; NIHSS 11+ as LVO), as NIHSS has been shown to have higher accuracy in separating nLVO and LVO than other stroke scales (Area Under the Receiver Operating Characteristic Curve = 0.86 \cite{duvekot_comparison_2021}). Applying this classification to the stroke admission data in England and Wales (Sentinel Stroke National Audit Programme), this provides an estimate of 30\% LVO and 70\% nLVO in the treatable population. This proportion split was applied to each of the LSOAs in the study (to divide the total SSNAP stroke admissions by LSOA into these two patient stroke type cohorts). These derived values were also cross-checked against stroke types identified in studies on pre-hospital selection of patients with suspected LVO \cite{de_la_ossa_herrero_design_2013}, where LVO made up 38\% of the population where RACE was applied. 

We used modified Rankin Scale (mRS) as a measure of outcome. mRS is the most commonly used instrument to describe post-stroke functional outcome \cite{quinn_functional_2009}, describing independence of living from a scale of 0 (no disability) through to 5 (severe disability), with death assigned an mRS of 6. We will use mRS 0-2 as a surrogate for independent living. Utility values for each mRS level were taken from Wang \textit{et al.} 
 \cite{wang_utility-weighted_2020}. The mean mRS score, mean utility and proportion of patients with mRS 0-2 in a given mRS distribution can be compared with those of a second mRS distribution (representing another treatment scenario) to find the mean added utility, mean change in mRS, and change in proportion with mRS 0-2 respectively.

Our model calculates the patients mRS outcome distribution  based on time to treatment for three patient-treatment cohorts: nLVO treated with IVT; LVO treated with IVT; and LVO treated with MT. For each patient-treatment cohort we derived an mRS distribution for treatment given at \emph{t=0} (time of stroke onset) and an mRS distribution for treatment given at \emph{t=No Effect} (time of no effect of treatment). Assuming that log odds fall linearly over time \cite{emberson_effect_2014, fransen_time_2016}, the resulting \textit{t=0} and \textit{t=No Effect} mRS distributions may be interpolated to calculate an mRS distribution for treatment at any given time.

We derived each mRS distribution for treatment at \emph{t=0} using the methodology of log-odds ratio of a good outcome falling linearly with time to treatment \cite{emberson_effect_2014, fransen_time_2016}. For each of the nLVO and LVO mRS distributions, we use the change of log-odds ratio of mRS 0-1 with time to IVT \cite{emberson_effect_2014} to scale the probability of mRS 0-1 from \emph{t=No Effect} to \emph{t=0}. We also require the mRS distributions for each patient cohort pre-stroke (sourced from SSNAP data) and if they received no treatment. The nLVO no-treatment population is the weighted difference of no-treatment populations containing both patients with nLVOs and LVOs \cite{lees_time_2010} and containing only LVO patients \cite{goyal_endovascular_2016}, where weights of 149\% and 49\% respectively result in the nLVO no-treatment probability of mRS 0-1 matching a reference population \cite{emberson_effect_2014}. The resulting proportions of LVO (49\%) and nLVO (51\%) patients are similar to those in clinical trials \cite{ist-3_collaborative_group_benefits_2012, emberson_effect_2014}. 
These \emph{t=0} mRS 0-1 data inform the weights for weighted averages of the pre-stroke and \emph{t=No Effect} mRS distributions that create the full \emph{t=0} mRS distributions (64.3\% and 35.7\% for nLVO, 25.5\% and 74.5\% for LVO respectively). The resulting \emph{t=No Effect} and \emph{t=0} mRS distributions for nLVO are consistent with the decline of chance of mRS 0-1 with time \cite{holodinsky_modeling_2018}. The MT \emph{t=0} mRS distribution is defined as the weighted average of 75\% of the pre-stroke and 25\% of the \emph{t=No Effect} distributions, following a reference rate of successful recanalisation \cite{hui_efficacy_2020}. The MT excess death rate is set to 4.0\% to ensure that log-odds falling linearly with time between the values of mRS 0-5 in the \emph{t=0} and in the \emph{t=No Effect} mRS distributions is consistent with a reference average mortality rate at the average MT time \cite{goyal_endovascular_2016}. The resulting \emph{t=0} and \emph{t=No Effect} mRS 0-2 probabilities give close matches to a reference decline of chance of mRS 0-2 with time \cite{fransen_time_2016}.


The time to no effect was 6.3 hours for IVT \cite{emberson_effect_2014} and 8.0 hours for MT \cite{ fransen_time_2016} (our model did not include selection of patients who may still benefit from treatment beyond these durations). We derived each \textit{No Effect} mRS distribution by applying the excess death rate due to treatment equally across the mRS distribution of patients who received no treatment. We calculate the remaining IVT data using excess death rates of 1.1\% for nLVO and 3.4\% for LVO, from the difference in death rates of trial groups given IVT and given no treatment \cite{emberson_effect_2014}. 

\subsection{Scenario analysis}

Scenario analysis was undertaken to investigate how changing assumed model parameters (the process durations, in minutes) affect outcomes across all LSOAs. The parameter values were varied according to the following, with all combinations modelled:

\begin{minipage}{1.0\textwidth}  % Define the width of the minipage
\begin{spacing}{1.2}
\begin{itemize}
    \item All patients:
    \begin{itemize}
        \item Stroke onset to call: 0, 60, 120, 180
    \end{itemize}
    \item Usual care:
    \begin{itemize}
        \item Call to ambulance arrival: 15, 30, 45
        \item Ambulance on-scene: 20, 30, 45
        \item Hospital arrival to IVT: 30, 45
        \item Transfer-related delay (excluding travel time): 30, 60, 90
        \item Hospital arrival to MT: 60, 90, 120
    \end{itemize}
    \item Mobile stroke units:
    \begin{itemize}
        \item Call to MSU dispatch: 0, 15, 30, 45
        \item MSU arrival to IVT: 15, 30, 45
        \item MSU on-scene post-IVT: 5, 15
        \item MSU hospital arrival to MT: 30, 60, 90
    \end{itemize}
\end{itemize}
\end{spacing}
\end{minipage}


\subsection{Geographic analysis}

To study geographic variation in benefit of MSUs, a single set of parameters was chosen, reflecting a reasonable base case for performance of usual care and MSU care. Process times (minutes) are shown below, with ambulance and MSU travel times and inter-hospital travel times dependent on patient location (LSOA). In this base case scenario MSUs are based at comprehensive stroke centres only. Admission numbers per LSOA were taken from HES 2017-2019 (using OCD-10 codes of I61, I63 and I64). Population density was taken from the Office of National Statistics 2011 census.


\begin{minipage}{1.0\textwidth}  % Define the width of the minipage
\begin{spacing}{1.2}
\begin{itemize}
    \item All patients:
    \begin{itemize}
        \item Stroke onset to call: 60
    \end{itemize}
    \item Usual care:
    \begin{itemize}
        \item Call to ambulance arrival: 20
        \item Ambulance on-scene: 30
        \item Hospital arrival to IVT: 45
        \item Transfer-related delay (excluding travel time): 60
        \item Hospital arrival to MT: 90
    \end{itemize}
    \item Mobile stroke units:
    \begin{itemize}
        \item Call to MSU dispatch: 15
        \item MSU arrival to IVT: 30
        \item MSU on-scene post-IVT: 5
        \item MSU hospital arrival to MT: 60
    \end{itemize}
\end{itemize}
\end{spacing}
\end{minipage}


\subsection{Varying number of MSU base locations}

In order to study the effect of changing the number of MSU base locations, a greedy algorithm was used. In this method, 100 MSU base locations are sequentially added, with each additional one chosen from either only CSCs, or from any stroke unit type, in the order of maximum improvement in outcomes (utility). The utility gain is calculated for those patients treated by an MSU rather than usual care; the algorithm is therefore looking at the effect of changing the number of MSU base locations, rather than the number of physical MSUs.